\documentclass[english]{article}
 \usepackage[utf8]{inputenc}
 \usepackage[T1]{fontenc}\usepackage{geometry}
\geometry{verbose,tmargin=3cm,bmargin=3cm,lmargin=2.5cm,rmargin=2.5cm}
\usepackage{verbatim}
\usepackage{float}
\usepackage{textcomp}
\usepackage{amstext}
\usepackage{amssymb}
\usepackage{graphicx}
\usepackage{esint}
\usepackage[numbers]{natbib}
\usepackage{lmodern} 

\makeatletter
%%%%%%%%%%%%%%%%%%%%%%%%%%%%%% User specified LaTeX commands.
\usepackage{ae,aecompl}
\usepackage{lineno}

\usepackage{setspace}
\doublespacing

\makeatother
\usepackage[main=french]{babel}

\begin{document}

\begin{abstract}

1 - La persistance de la biodiversit\'e du phytoplancton malgr\'e la comp\'etition pour les ressources est depuis longtemps une source de questionnement pour les \'ecologues. Afin d'identifier, parmi les nombreux m\'ecanismes de coexistence sugg\'er\'es par la th\'eorie et les exp\'erimentations, ceux qui maintiennent la diversit\'e dans les \'ecosyst\`emes naturels, des \'etudes à long-terme sont essentielles. \\

2 - Nous avons analys\'e des s\'eries temporelles longues de comptages de phytoplancton. Le phytoplancton a \'et\'e identifi\'e au niveau du genre, toutes les deux semaines pendant vingt ans sur dix sites, le long du littoral français. Des mod\`eles autor\'egressifs multivari\'es ont permis d'inférer des r\'eseaux d'interactions biotiques, tout en prenant en compte les variables abiotiques susceptibles d'affecter les fluctuations du phytoplancton. Nous avons ensuite calcul\'e le ratio entre les interactions intra et inter-taxa (une mesure de la diff\'erentiation des niches), la fr\'equence des interactions positives et n\'egatives ainsi que la façon dont les m\'etriques de stabilit\'e (au niveau du r\'eseau et du genre) sont li\'ees \`a la complexit\'e du r\'eseau et \`a l'autor\'egulation ou l'abondance de chaque genre. \\

3 - Nous avons montr\'e qu'une autor\'egulation forte, c'est \`a dire une force de comp\'etition au sein d'un même taxon (ici, le genre) un ordre de grandeur sup\'erieure aux forces d'interaction entre taxa, \'etait omnipr\'esente dans les r\'eseaux d'interactions phytoplanctoniques. Cette comp\'etition intragenre beaucoup plus forte sugg\`ere que la diff\'erentiation des niches, plutôt que la neutralit\'e, est fréquente pour le phytoplancton. En outre, les r\'eseaux d'interactions \'etaient domin\'es par des effets nets positifs d'un taxon sur un autre (en moyenne, plus de la moiti\'e des interactions \'etaient positives). Bien que la stabilit\'e (au sens de la r\'esilience des r\'eseaux) ne soit pas li\'ee aux mesures de complexit\'e, nous avons r\'ev\'el\'e des relations entre l'autor\'egulation, la force des interactions entre genres et l'abondance. Les genres les moins communs tendent \`a être plus fortement autor\'egul\'es et peuvent ainsi se maintenir malgr\'e la comp\'etition avec les genres les plus abondants. \\

4 - \textit{Synth\`ese} : Nous d\'emontrons qu'une diff\'erentiation des niches, une facilitation fréquente parmi les taxa de phytoplancton, et des covariances stabilisant les forces d'interaction, devraient être des facteurs relativement communs dans les communit\'es phytoplanctoniques qui coexistent in situ. Ces propri\'et\'es structurelles peuvent donc être attendues de mod\`eles m\'ecanistes multi-taxa pour le phytoplancton. Nous discutons des m\'ecanismes, tels que la pr\'edation ou des mouvements limités à l'échelle microscopique, qui pourraient expliquer ces résultats, et ainsi ouvrir de nouvelles pistes de recherche. 

\end{abstract}

\end{document}
