\documentclass[english]{article}
 \usepackage[utf8]{inputenc}
 \usepackage[T1]{fontenc}\usepackage{geometry}
\geometry{verbose,tmargin=3cm,bmargin=3cm,lmargin=2.5cm,rmargin=2.5cm}
\usepackage{verbatim}
\usepackage{float}
\usepackage{textcomp}
\usepackage{amstext}
\usepackage{amssymb}
\usepackage{graphicx}
\usepackage{esint}
\usepackage[numbers]{natbib}
\usepackage{lmodern} 

\makeatletter
%%%%%%%%%%%%%%%%%%%%%%%%%%%%%% User specified LaTeX commands.
\usepackage{ae,aecompl}
\usepackage{lineno}

\usepackage{setspace}
\doublespacing

\makeatother
\usepackage[main=french]{babel}

\begin{document}

\begin{abstract}

1 - La persistance de la biodiversit\'e du phytoplancton malgr\'e la comp\'etition pour des ressources \'el\'ementaires est depuis longtemps une source de questionnement pour les \'ecologues. Afin d'identifier, parmi les nombreux m\'ecanismes de coexistence sugg\'er\'es par la th\'eorie et les exp\'erimentations, ceux qui maintiennent r\'eellement la diversit\'e dans les \'ecosyst\`emes naturels, des \'etudes de long-terme sont essentielles. \\

2 - Nous avons analys\'e un grand jeu de donn\'ees comprenant des s\'eries temporelles d'abondances phytoplanctoniques grâce \`a des mod\`eles autor\'egressifs multivari\'es dynamiques. Le phytoplancton a \'et\'e compt\'e et identifi\'e au niveau du genre, toutes les deux semaines pendant vingt ans sur dix sites le long du littoral français. Les mod\`eles autor\'egressifs multivari\'es ont permis d'estimer des r\'eseaux d'interactions biotiques, tout en prenant en compte les variables abiotiques qui sont susceptibles de contrôler les fluctuations du phytoplancton. Nous avons ensuite calcul\'e le ratio entre les interactions intra et inter-taxa (une mesure de l'autor\'egulation, i.e. de la diff\'erentiation de niches), la fr\'equence des interactions positives et n\'egatives ainsi que la façon dont les m\'etriques de stabilit\'e (au niveau du r\'eseau et du genre) sont li\'ees \`a la complexit\'e du r\'eseau et \`a l'autor\'egulation ou l'abondance de chaque genre. \\

3 - Nous avons montr\'e qu'une autor\'egulation forte, c'est \`a dire une force de comp\'etition au sein d'un même taxon (ici, le genre) un ordre de grandeur sup\'erieur aux interactions entre taxa, \'etait omnipr\'esente dans les r\'eseaux d'interactions phytoplanctoniques. Cette comp\'etition intragenre beaucoup plus forte sugg\`ere que la diff\'erentiation de niches, plutôt que la neutralit\'e, est usuelle pour le phytoplancton. En outre, les r\'eseaux d'interactions \'etaient domin\'es par des effets nets positifs d'un taxon sur un autre (en moyenne, plus de la moiti\'e des interactions \'etaient positives). Bien que la stabilit\'e (au sens de la r\'esilience des r\'eseaux) ne soit pas li\'ee aux mesures de complexit\'e, nous avons r\'ev\'el\'e des relations entre l'autor\'egulation, la force des interactions intergenre et l'abondance. Les genres les moins communs tendent \`a être plus fortement autor\'egul\'es et peuvent ainsi se maintenir malgr\'e la comp\'etition avec les genres les plus abondants. \\

4 - \textit{Synth\`ese} : Nous d\'emontrons qu'une diff\'erentiation de niches, une facilitation fréquente parmi les taxa de phytoplancton, et des covariances stabilisant les forces d'interaction, devraient être des facteurs communs pour la coexistence des communit\'es phytoplanctoniques dans les milieux naturels. Ces propri\'et\'es structurelles sont attendus dans des mod\`eles m\'ecanistes plausibles des communaut\'es de phytoplancton. Nous discutons des m\'ecanismes qui peuvent corroborer ces r\'esultats, tels que la pr\'edation ou la restriction des mouvements \`a micro\'echelle, ce qui ouvre la voie \`a des recherches futures. 

\end{abstract}

\end{document}
